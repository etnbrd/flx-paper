\begin{abstract}

The development of a web application often starts with a feature-oriented approach allowing to quickly react to users feedbacks.
However, this approach poorly scales in performance.
Yet, the audience of a web application can increase by an order of magnitude in a matter of hours. This first approach is unable to deal with the higher connections spikes.
It leads the development team to adopt a scalable approach often linked to new development paradigm such as dataflow programming.
This represent a disruptive and continuity-threatening shift of technology.
To avoid this shift, we propose to abstract the feature-oriented development into a high-level language, allowing a high-level code reasoning.
This reasoning allows code mobility so as to dynamically cope with audience growth and decrease.

We propose a compiler that transforms a Javascript, monolithic, web application into a network of small independent parts communicating by message streams.
We evaluate the approach by applying this compiler to a real web application.
We succssefully transform a web application to parallelize the execution of an independent part.
We named these parts \textit{fluxions}, by contraction between a flux and a function.
The dynamic reorganization of these parts in a cluster of machine can help an application to deal with its load in a similar way network routers do with IP traffic.

\end{abstract}