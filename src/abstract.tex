\begin{abstract}

The audience's growth of a web application is highly uncertain, it can increase and decrease in a matter of hours if not minutes.
This uncertainty often leads the development team to quickly adopt disruptive and continuity-threatening shifts of technology to deal with the higher connections spikes.
To avoid these shifts, we propose to abstract web applications into a high-level language, allowing a high-level code reasoning.
That reasoning may allow one to provide code mobility to dynamically cope with audience growth and decrease.

We think a web applications can be expressed as a network of small autonomous parts moving from one machine to another and communicating by message streams.
We named these parts \textit{fluxions}, by contraction between a flux and a function.
\textit{Fluxions} are distributed over a network of machines according to their interdependencies to minimize overall data transfers.
We expect that this dynamic reorganization can allow an application to deal with its load.

Our high-level language proposition consists of a compiler and its execution model to express a web application into a network of \textit{fluxions} and streams.
\end{abstract}