\begin{abstract}

The audience's growth of a web application is highly uncertain, it can increase and decrease in a matter of hours if not minutes.
This uncertainty often leads the development team to quickly adopt disruptive and continuity-threatening shifts of technology to handle the higher connections spikes.
% The audience's growth a web application needs to adapt to, often leads its development team to quickly adopt disruptive and continuity-threatening shifts of technology.
To avoid these shifts, we propose an approach that abstracts web applications into a high-level language, allowing code mobility to dynamically cope with audience growth and decrease.

We think a web application can be depicted as a network of small autonomous parts moving from one machine to another and communicating by message streams.
The high-level language we propose aims to express these parts and their streams.
We named these parts \textit{fluxions}, by contraction between a flux and a function.
\textit{Fluxions} are distributed over a network of machines according to their interdependencies to minimize overall data transfers.
We expect that this dynamic reorganization can allow an application to handle its load.

Our high-level language proposition consists of an execution model which dynamically adapts to the execution environment, and a tool to automate the technological shift between the classical model and the proposed one.

\end{abstract}