\begin{abstract}

The audience's growth of a web applicaiton is very incertain, it can increase and deacrease in a matter of hours if not minutes.
This incertain growth often leads the development team to quickly adopt disruptive and continuity-threatening shifts of technology.
% The audience's growth a web application needs to adapt to, often leads its development team to quickly adopt disruptive and continuity-threatening shifts of technology.
To avoid these shifts, we propose an approach that abstracts web applications into an high-level language, which authorizes code mobility to cope with audience dynamic growth and decrease.

We think a web application can be depicted as a network of small autonomous parts moving from one machine to another and communicating by message streams.
The high-level language we propose aims at expressing these parts and their streams.
We named these parts \textit{fluxions}, by contraction between a stream\footnote{flux in french} and a function.
\textit{Fluxions} are distributed over a network of machines according to their interdependencies to minimize overall data transfers.
We expect that this dynamic reorganization can allow an application to cope with its load.

Our high-level language proposal consists of an execution model which dynamically adapts itself to the execution environment, and a tool to automate the technological shift between the classical model and the proposed one.

\end{abstract}