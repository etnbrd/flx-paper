\section{Real case test} \label{section:evaluation}

The goal of this evaluation is to prove the possibility for an application to be compiled into a network of independent parts.
We want to show the current limitation of this isolation and the modification needed on the application to circumvent these limitations.
% TODO -> and finally, we present the possibilities of future works.
% We want to show the limitations of this isolation for future works, and the modifications needed to circumvent these limitations.

We present a test of our compiler on a real application, gifsockets-server\ftnt{https://github.com/twolfson/gifsockets-server}.
% This application is part of the selection from our previous paper \cite{Brodu2015}.
% We chosed it because it is a working application, simple enough to illustrate this evaluation.
This application was selected from the \texttt{npm} registry because it depends on \texttt{express}, it is tested and working, and it simple enough to illustrate this evaluation.
It is part of the selection from a previous work. %\cite{Brodu2015}.

This application is a real-time chat using gif-based communication channels.
% The client sends a request containing a text typed by the user.
The server transforms the received text into a gif frame, and pushes it back to a never-ending gif to be displayed on the client.
Listing \ref{lst:gifsocket} is a simplified version of this application, containing only essential lines.

\begin{code}[js, caption={Simplified version of gifsockets-server},label={lst:gifsocket}]
var express = require('express'),
    app = express(),
    routes = require('gifsockets-middleware'), //@\label{lst:gifsocket:gif-mw}@
    getRawBody = require('raw-body');

function bodyParser(limit) { //@\label{lst:gifsocket:bodyParser}@
  return function saveBody(req, res, next) { //@\label{lst:gifsocket:saveBody}@
    getRawBody(req, { //@\label{lst:gifsocket:getRawBody}@
      expected: req.headers['content-length'],
      limit: limit
    }, function (err, buffer) { //@\label{lst:gifsocket:callback}@
      req.body = buffer;
      next(); //@\label{lst:gifsocket:next}@
    });
  };
}

app.post('/image/text', bodyParser(1 * 1024 * 1024), routes.writeTextToImages); //@\label{lst:gifsocket:app.post}@
app.listen(8000);
\end{code}

% The web application framework used in this application, \textit{express}, allows to register chains of functions to process user requests.
On line \ref{lst:gifsocket:app.post}, the application registers two functions to process the requests received on the url \texttt{/image/text}.
The closure \texttt{saveBody}, line \ref{lst:gifsocket:saveBody}, returned by \texttt{bodyParser}, line \ref{lst:gifsocket:bodyParser}, and the method \texttt{routes.write\-Text\-To\-Images} from the external module \texttt{gifsockets-middleware}, line \ref{lst:gifsocket:gif-mw}.
The closure \texttt{saveBody} calls the asynchronous function \texttt{getRawBody} to get the request body.
Its callback handles the errors, and calls \texttt{next} to continue the execution with the next function, \texttt{routes.write\-Text\-To\-Images}.

% The closure \texttt{saveBody} gather the whole request, and let \textit{express} call the next function in the chain, \texttt{routes.write\-Text\-To\-Images}, by calling \texttt{next}, line \ref{lst:gifsocket:next}.


\subsection{Compilation}

We compile this application with the compiler detailed in section \ref{section:compiler}.
% The function \texttt{getRawBody}, line \ref{lst:gifsocket:getRawBody}, is asynchronous.
The function call \texttt{app.post}, line \ref{lst:gifsocket:app.post}, is a rupture point.
However, its callbacks, \texttt{bodyParser} and \texttt{routes.write\-Text\-To\-Images} are evaluated as function only at runtime.
For this reason, the compiler ignores this rupture point, to avoid interfering with the evaluation.
% For future works, we intend to improve the compiler with a runtime part, to detect callbacks dynamically evaluated.
% \comment{TODO move this sentence into a future works paragraph, or explain more here ?}

The compilation result is in listing \ref{lst:flx-gifsocket}.
The compiler detects a rupture point : the function \texttt{getRawBody} and its anonymous callback, line \ref{lst:gifsocket:callback}.
It encapsulates this callback in a fluxion named \texttt{anonymous\-\_1000}.
The callback is replaced with a stream placeholder to send the message stream to this downstream fluxion.
The variables \texttt{req}, and \texttt{next} are append to this message stream, to propagate their value from the \texttt{main} fluxion to the \texttt{anonymous\-\_1000} fluxion.
% These variables are declared in the \texttt{main} fluxion, therefore, it adds them in the stream to the downstream fluxion \texttt{anonymous\-\_1000}.

% The compilation result needs to be modified manually to fix some mistakes.
% These mistakes comes from the compiler being unstable, and in early stages of development, but most of these mistakes could be avoided in the future.
% The modified and simplified compilation result is in listing \ref{lst:flx-gifsocket}.

When \texttt{anonymous\-\_1000} is not isolated from the \texttt{main} fluxion, the compilation result works as expected.
The variables used in the fluxion, \texttt{req} and \texttt{next}, are still shared between the two fluxions.
Our goal is to isolate the two fluxions, to be able to safely parallelize their executions.

\begin{code}[flx, caption={Compilation result of gifsockets-server},label={lst:flx-gifsocket}]
flx main
>> anonymous_1000 [req, next]
  var express = require('express'),
      app = express(),
      routes = require('gifsockets-middleware'), //@\label{lst:flx-gifsocket:gif-mw}@
      getRawBody = require('raw-body');

  function bodyParser(limit) { //@\label{lst:flx-gifsocket:bodyParser}@
    return function saveBody(req, res, next) { //@\label{lst:flx-gifsocket:saveBody}@
      getRawBody(req, { //@\label{lst:flx-gifsocket:getRawBody}@
        expected: req.headers['content-length'], //@\label{lst:flx-gifsocket:req.headers}@
        limit: limit
      }, >> anonymous_1000);
    };
  }

  app.post('/image/text', bodyParser(1 * 1024 * 1024), routes.writeTextToImages); //@\label{lst:flx-gifsocket:app.post}@
  app.listen(8000);

flx anonymous_1000
-> null
  function (err, buffer) { //@\label{lst:flx-gifsocket:callback}@
    req.body = buffer; //@\label{lst:flx-gifsocket:buffer}@
    next(); //@\label{lst:flx-gifsocket:next}@
  }
\end{code}

\subsection{Isolation}

In listing \ref{lst:flx-gifsocket}, the fluxion \texttt{anonymous\_1000} modifies the object \texttt{req}, line \ref{lst:flx-gifsocket:buffer}, to store the text of the received request,and it calls \texttt{next} to continue the execution, line \ref{lst:flx-gifsocket:next}.
These operations produce side-effects that should propagate in the whole application, but the isolation prevent this propagation.
Isolating the fluxion \texttt{anonymous\_1000} produces runtime exceptions.
We detail in the next paragraph, how we handle this situation to allow the application to be compiled.
This evaluation highlights the current limitations of the compiler, and present future works around them.

\subsubsection{Variable \texttt{req}}

The variable \texttt{req} is read in fluxion \texttt{main}, lines \ref{lst:flx-gifsocket:getRawBody} and \ref{lst:flx-gifsocket:req.headers}.
Then it is associated in fluxion \texttt{anonymous\_1000} to \texttt{buffer}, line \ref{lst:flx-gifsocket:buffer}.
The compiler is unable to identify further usages of this variable.
However, the side effect resulting from this association impacts a variable in the scope of the next callback, \texttt{routes.writeTextToImages}.
We modified the application to explicitly propagate this side-effect to continue the execution with the function \texttt{next}.

% For future works, instead of relying only on the source code, we intend to analyze the memory deeper to detect such side-effects.
% \comment{TODO move this sentence into a future works paragraph, or explain more here ?}

\subsubsection{Closure \texttt{next}}

The function \texttt{next} is a closure provided by the \texttt{Router} to continue the execution with the next function to handle the client request.
Because it indirectly relies on network sockets, it is impossible to isolate its execution with the \texttt{anonymous\-\_1000} fluxion.
Instead, we modify \textit{express}, so as to be compatible with the fluxionnal execution model.
We explain the modification below.% , and illustrate them in listing \ref{lst:mflx-gifsocket}.
% The \texttt{req} and \texttt{next} objects needs to stay on the master worker to preserve their closures.

Originally, the function \texttt{next} is the continuation to allow the anonymous callback, to continue the execution with the next function to handle the request.
To isolate the anonymous callback, this function is replaced on both ends.
The \textit{express} \texttt{Router} register a fluxion named \texttt{express\_dispatcher} to continue the execution after the fluxion \texttt{anonymous\-\_1000}.
This fluxion is in the same group as the \texttt{main} fluxion, hence it has access to network sockets and to the original variable \texttt{req}.
The original \texttt{next} function in the anonymous callback is replaced by a placeholder to push the stream to the fluxion \texttt{express\_dispatcher}.
The fluxion \texttt{express\_dispatcher} receives the stream from the upstream fluxion \texttt{anonymous\-\_1000}, merges back the modification in the variable \texttt{req}, before calling the original function \texttt{next} to continue the execution.

%  to holds these objects on the master worker, and receives the result of the isolated fluxion \texttt{anonymous\_1000}.


% The application sends the original object to the fluxion \texttt{express\_dispatcher} and serialized copies to the isolated fluxion \texttt{anonymous\_1000}.
% In this latter fluxion, the anonymous callback do its computation ; it assigns the received \texttt{body} as an attribute of \texttt{req}.

% In the original application, the anonymous callback finishes by calling the function \texttt{next} to let the \texttt{Router} call the next function to process the request.
% In the compiled application, this function \texttt{next} is not available on the isolated worker.
% Instead, the anonymous callback inside \texttt{anonymous\-\_1000} calls a function \texttt{next} specially provided by the fluxionnal execution model to send a message to the fluxion \texttt{express\-\_dispatcher} with the modified copies of \texttt{req} and \texttt{res}.
% and call the original function \texttt{next} on the master worker.

% In the original application, \textit{express} relies on side-effects on the objects \texttt{req} and \texttt{res} to get their modifications.
% The call to \texttt{next} doesn't need them as argument.
% In the isolated fluxion, as the serialized object and their originals are isolated from each other, side-effects don't propagate.
% The special \texttt{next} function needs explicit references to the modified objects to send them back to \texttt{express\_dispatcher}.
% The fluxion \texttt{express\_dispatcher} then merges back the modified copies and their originals, before calling the original function \texttt{next}.

% \begin{code}[flx, caption={Simplified modification on the compiled result},label={lst:mflx-gifsocket}]
% flx main
% >> anonymous_1000 [req, next]
%   var express = require('express'),
%       app = express(),
%       routes = require('gifsockets-middleware'), //@\label{lst:flx-gifsocket:gif-mw}@
%       getRawBody = require('raw-body');

%   function bodyParser(limit) { //@\label{lst:flx-gifsocket:bodyParser}@
%     return function saveBody(req, res, next) { //@\label{lst:flx-gifsocket:saveBody}@
%       getRawBody(req, { //@\label{lst:flx-gifsocket:getRawBody}@
%         expected: req.headers['content-length'], //@\label{lst:flx-gifsocket:req.headers}@
%         limit: limit
%       }, >> anonymous_1000);
%     };
%   }

%   app.post('/image/text', bodyParser(1 * 1024 * 1024), routes.writeTextToImages); //@\label{lst:flx-gifsocket:app.post}@
%   app.listen(8000);

% flx anonymous_1000
% -> express_dispatcher
%   function (err, buffer) { //@\label{lst:flx-gifsocket:callback}@
%     req.body = buffer; //@\label{lst:flx-gifsocket:buffer}@
%     -> express_dispatcher; //@\label{lst:flx-gifsocket:next}@
%   }

% flx express_dispatcher
% -> null
%   merge(req, msg.req);
%   next();
% \end{code}

After the modifications detailed above, the server works as expected for the subset of functionalities we modified.
The isolated fluxion correctly receives, and returns its serialized messages.
The client successfully receives a gif frame containing the text.

\subsection{Future works}

We intend to implement the compilation presented into a runtime.
A just-in-time compiler allows to detect callbacks dynamically evaluated, and to analyze the memory to identify side-effects propagations, instead of relying only on the source code.
This memory analysis allows closure serialization required to compile application using higher-order functions.

% \subsubsection{Fluxionnal web framework}

% In case of error, the anonymous callback calls \texttt{res.writeHead} and \texttt{res.end}.
% These two closures are similar to the closure \texttt{next}.
% It is possible to extend the modifications presented above to build a complete web application framework, with some limitations detailed below.
% Indeed, the evaluation proves that it is possible to modify the \textit{express} framework to be compatible with the fluxionnal execution model.

% The closure \texttt{next} is assured to be called only once at the end of the callback.
% It can be called asynchronously, and can be assimilated to a rupture point.
% Therefore, it is safe to replace it by a communication between the two workers.
% On the other hand, the functions \texttt{res.writeHead} and \texttt{res.end} are synchronous.
% It is unsafe to replace every call by a communication between the two workers.
% It would lead to race conditions.
% These calls needs to modify the serialized, local copies of \texttt{req} and \texttt{res}, and sends the result to the master only once.