\section{Evaluation} \label{section:evaluation}

The goal of this evaluation is to prove the possibility for an application to be compiled in order to defer parts of its execution on a remote worker.
We want to show the limitations of this isolation for future works, and the modifications needed to circumvent these limitations.

For brevity, we present in this paper only one test on a real application, gifsockets-server\ftnt{https://github.com/twolfson/gifsockets-server}.
This application is part of the selection from our previous paper \cite{Brodu2015}.
We chosed it because it is a complete, working, application, not a library, and it is simple enough to illustrate this evaluation.

This application is an example of a chat using gif-based communication channels.
The client, a page containing a never-ending gif, sends a request containing a text typed by the user.
The server transforms this text into a gif frame, and pushes this frame back to the never-ending gif to be displayed.
Listing \ref{lst:gifsocket} is a simplified version of this application, containing only critical sections.

The web application framework used in this application, \textit{express}, allows to register chains of functions to process user requests.
On line \ref{lst:gifsocket:app.post}, the call to \texttt{app.post} register two functions to process the requests on the url \texttt{/image/text}.
The closure \texttt{saveBody}, line \ref{lst:gifsocket:saveBody}, returned by \texttt{bodyParser}, line \ref{lst:gifsocket:bodyParser}, and the method \texttt{routes.write\-Text\-To\-Images} from the external module \texttt{gifsockets-middleware}, line \ref{lst:gifsocket:gif-mw}.
The closure \texttt{saveBody} gather the whole request, and let \textit{express} call the next function in the chain, \texttt{routes.write\-Text\-To\-Images}, by calling \texttt{next}, line \ref{lst:gifsocket:next}.

\begin{code}[js, caption={Simplified version of gifsockets-server},label={lst:gifsocket}]
var express = require('express'),
    app = express(),
    routes = require('gifsockets-middleware'); //@\label{lst:gifsocket:gif-mw}@
    getRawBody = require('raw-body');

function bodyParser(limit) { //@\label{lst:gifsocket:bodyParser}@
  return function saveBody(req, res, next) { //@\label{lst:gifsocket:saveBody}@
    getRawBody(req, { //@\label{lst:gifsocket:getRawBody}@
      expected: req.headers['content-length'],
      limit: limit
    }, function (err, buffer) { //@\label{lst:gifsocket:callback}@
      // If there was an error (e.g. bad length, over length), respond poorly
      if (err) {
        res.writeHead(500, {
          'content-type': 'text/plain'
        });
        return res.end('Content was too long');
      }
      req.body = buffer;
      next(); //@\label{lst:gifsocket:next}@
    });
  };
}

app.post('/image/text', bodyParser(1 * 1024 * 1024), routes.writeTextToImages); //@\label{lst:gifsocket:app.post}@
app.listen(8000);
\end{code}


\subsection{Compilation}

We compile this application with the compiler detailed in section \ref{section:compiler}.
% The function \texttt{getRawBody}, line \ref{lst:gifsocket:getRawBody}, is asynchronous.
The function call \texttt{app.get} is asynchronous, but the compiler is unable to detect the function \texttt{saveBody} returned by \texttt{bodyParser} as a callback.
The compiler detects only one rupture point, between \texttt{getRawBody} and its anonymous callback, line \ref{lst:gifsocket:callback}.
It encapsulates this callback in a fluxion named \texttt{anonymous\-\_1000}.
The original callback is replaced with a placeholder function to send a message to this fluxion now containing the callback.

The compiler identifies that this callback uses the variables \texttt{req}, \texttt{res}, and \texttt{next}.
It puts these variables in the stream of the message to send to the downstream fluxion \texttt{anonymous\-\_1000}.

The compilation result needs to be modified manually to fix some mistakes.
These mistakes comes from the compiler being unstable, and in early stages of development, but most of these mistakes could be avoided in the future.
The modified and simplified compilation result is in listing \ref{lst:flx-gifsocket}.

The compilation doesn't seem to introduce bugs, as the result of compilation executes without errors, and works as expected.
However, it is important to note that the fluxion \texttt{anonymous\_1000} is not yet isolated on a remote worker.
Therefore, the variables used in the fluxion, \texttt{req}, \texttt{res} and \texttt{next}, are still shared with the rest of the application.
Our goal is to isolate this fluxion in a different memory heap, to be able to safely parallelize its execution.

\begin{code}[flx, caption={Compilation result of the simplified version of gifsockets-server},label={lst:flx-gifsocket}]
flx app_js
>> anonymous_1000 [req, res, next]
  var express = require('express'),
      app = express(),
      routes = require('gifsockets-middleware'); //@\label{lst:flx-gifsocket:gif-mw}@
      getRawBody = require('raw-body');

  function bodyParser(limit) { //@\label{lst:flx-gifsocket:bodyParser}@
    return function saveBody(req, res, next) { //@\label{lst:flx-gifsocket:saveBody}@
      getRawBody(req, { //@\label{lst:flx-gifsocket:getRawBody}@
        expected: req.headers['content-length'],
        limit: limit
      }, >> anonymous_1000);
    };
  }

  app.post('/image/text', bodyParser(1 * 1024 * 1024), routes.writeTextToImages); //@\label{lst:flx-gifsocket:app.post}@
  app.listen(8000);

flx anonymous_1000
-> null
  function (err, buffer) { //@\label{lst:flx-gifsocket:callback}@
    // If there was an error (e.g. bad length, over length), respond poorly
    if (err) {
      res.writeHead(500, { //@\label{lst:flx-gifsocket:writeHead}@
        'content-type': 'text/plain'
      });
      return res.end('Content was too long');  //@\label{lst:flx-gifsocket:end}@
    }
    req.body = buffer; //@\label{lst:flx-gifsocket:buffer}@
    next(); //@\label{lst:flx-gifsocket:next}@
  }
\end{code}

\subsection{Isolation}

The variables \texttt{req} and \texttt{res} points to objects containing closures, and the variable \texttt{next} points to a closure.
The fluxion \texttt{anonymous\_1000} require access over these closures.
Indeed, in listing \ref{lst:flx-gifsocket}, it modifies the attribute \texttt{body} of the object \texttt{req}, line \ref{lst:flx-gifsocket:buffer}, to store the text of the received request.
It calls \texttt{next} to continue the execution, line \ref{lst:flx-gifsocket:next}.
And in case of error, it uses the function \texttt{res.writeHead}, line \ref{lst:flx-gifsocket:writeHead}, and \texttt{res.end}, line \ref{lst:flx-gifsocket:end}.
It is impossible to serialize closures from within Javascript.
Isolating the fluxion \texttt{anonymous\_1000} produces runtime exceptions because it lacks access to these closures.
We detail in the next paragraph, how the compiler handles this situation.
In this evaluation, we ignore the case of error, and focus on the closure \texttt{next}.

\subsubsection{next}

The function \texttt{next} is a closure over the \textit{express} \texttt{Router}.
This function is provided by the \texttt{Router} itself to allow one function in the chain to call the next.
It is impossible to send this closure to the isolated fluxion.
Instead, we modify \textit{express}, so as to be compatible with the fluxionnal execution model.

The \texttt{req}, \texttt{res} and \texttt{next} objects needs to stay on the master worker to preserve their closures.
The \textit{express} \texttt{Router} register a local fluxion named \texttt{express\_dispatcher} to holds these objects on the master worker, and receives the result of the isolated fluxion \texttt{anonymous\_1000}.
The application sends the original object to the fluxion \texttt{express\_dispatcher} and serialized copies to the isolated fluxion \texttt{anonymous\_1000}.
In this latter fluxion, the anonymous callback do its computation ; it assigns the received \texttt{body} as an attribute of \texttt{req}.

In the original application, the anonymous callback finishes by calling the function \texttt{next} to let the \texttt{Router} call the next function to process the request.
In the compiled application, this function \texttt{next} is not available on the isolated worker.
Instead, the anonymous callback inside \texttt{anonymous\-\_1000} calls a function \texttt{next} specially provided by the fluxionnal execution model to send a message to the fluxion \texttt{express\-\_dispatcher} with the modified copies of \texttt{req} and \texttt{res}.
% and call the original function \texttt{next} on the master worker.

In the original application, \textit{express} relies on side-effects on the objects \texttt{req} and \texttt{res} to get their modifications.
The call to \texttt{next} doesn't need them as argument.
In the isolated fluxion, as the serialized object and their originals are isolated from each other, side-effects don't propagate.
The special \texttt{next} function needs explicit references to the modified objects to send them back to \texttt{express\_dispatcher}.
The fluxion \texttt{express\_dispatcher} then merges back the modified copies and their originals, before calling the original function \texttt{next}.

After the modifications detailed above, the server works as expected for the subset of functionalities we modified.
The isolated fluxion correctly receives, and returns its serialized messages.
The client successfully receives a gif frame containing the text.
However, in this evaluation, we ignored the case of error.

\subsubsection{fluxionnal web framework}

In case of error, the anonymous callback calls \texttt{res.writeHead} and \texttt{res.end}.
These two closures are similar to the closure \texttt{next}.
It is possible to extend the modifications presented above to build a complete web application framework, with some limitations detailed below.
Indeed, the evaluation proves that it is possible to modify the \textit{express} framework to be compatible with the fluxionnal execution model.

The closure \texttt{next} is assured to be called only once at the end of the callback.
It can be called asynchronously, and can be assimilated to a rupture point.
Therefore, it is safe to replace it by a communication between the two workers.
On the other hand, the functions \texttt{res.writeHead} and \texttt{res.end} are synchronous.
It is unsafe to replace every call by a communication between the two workers.
It would lead to race conditions.
These calls needs to modify the serialized, local copies of \texttt{req} and \texttt{res}, and sends the result to the master only once.