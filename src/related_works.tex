\section{Related Works} \label{section:related}

\comment{
  Most of the code transformation intended for parallelization require annotations, or transform loops inside of a sequential program.
  Our approach is different in that we aim at avoiding completely annotations.
  Indeed, the knowledge of asynchronism from the developer is enough so that we don't need more annotations.
  Furthermore, the transformation of the loops in sequential program is inherently limited by the sequential surrounding.
  Amdahl's law state that such program cannot expect high speedup due to parallelization.
  On the otherhand, the event-loop our approach is based on doesn't have any sequential surrounding.
  It is the most outer loops, because it is not part of the language, but of the runtime.
  Therefore, our approach is not limited by this sequential surrounding.
}



The idea to split a task into independent parts goes back to the Actor's model \cite{Hewitt1973} in 1973, and to Functional programming, like Lucid \cite{Ashcroft1977} in 1977 and all the following works on DataFlow leading up to Flow-Based programming (FBP) and Functional Reactive Programming (FRP).
Both FBP and FRP, recently got some attention in the Javascript community with the projects \textit{NoFlo}\ftnt{http://noflojs.org/}, \textit{Bacon.js}\ftnt{https://baconjs.github.io/} and \textit{react}\ftnt{https://facebook.github.io/react/}.

The execution model we presented in section \ref{section:model}, is inspired by some works on scalability for very large system, like the Staged Event-Driven Architecture (SEDA) by Matt Welsh \cite{Welsh2000}, System S developped in the IBM T. J. Watson research center \cite{Jain2006,Wu2007}, and later the MapReduce architecture \cite{Dean2008}.
It also drew its inspiration from more recent work following SEDA.
Among the best-known following works, we cited in the introduction Spark \cite{Zaharia2010, Zaharia2012}, MillWheel \cite{Akidau2013}, Timestream \cite{Qian2013} and Storm \cite{Toshniwal2014}.
The first part of our work stands upon these thorough studies.
However, we believe that it is too difficult for common developers to express their algorithm into a network of independent parts communicating through messages.
This belief motivated us to propose a compiler from an imparative programming model to these more scalable, distributed execution engines.

% The compiler makes use of the work on the BASE\cite{Fox1997} data semantics to justify trading off consitency for availability.
% We cite the work on the BASE\cite{Fox1997} and ACID data semantics, but the compiler doesn't use these works.

The transformation of an imperative programming model to be executed onto a parallel execution engine was recently addressed by Fernandez \textit{et. al.} \cite{Fernandez2014a}.
However, like in similar works \cite{Mitchell2012,Power2010}, the developer needs to manually specify the distribution of state.
We believe the difficulties encountered by developers in concurrent programming models lies more in the distribution of states, than on the structuration of the algorithm.
Indeed, developers seems to have little difficulties programming in an asynchronous concurrent programming model, like the Javascript event-loop.
In such programming model, the memory is global but the algorithm is ripped into multiple, asynchronous steps.
While the synchronous concurrent programming model, based on multi-threading and locks to assure the consistency of shared states, is known to be more difficult to apprehend by novice developers \cite{Adya2002}.


% TODO
% There is some work on the transformation of a program into distributed parts\cite{Amini2012, Petit2009}.

% Promises\cite{Liskov1988} are related to our work as they are abstractions from a concurrent programming style, to an asynchronous and parallel execution model.
% However, our approach using Node.js callback asynchronism to automate this abstraction seems unexplored yet.

Our compiler uses the \textit{estools} suite to parse, manipulate and generate source code from Abstract Syntax Tree (AST)\ftnt{https://github.com/estools}.
It modifies AST, as described in \cite{Jones2003}.
The implementation of the analyzer might be inspired from the points-to analysis in future works \cite{Wei2014}.
Our implementation is based on the work by Ryan Dahl : \textit{Node.js}\ftnt{https://nodejs.org/}, as well as on one of the best-known \textit{Node.js} web framework : \textit{Express}\ftnt{http://expressjs.com/}.