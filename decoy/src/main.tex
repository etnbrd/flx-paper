\documentclass{sig-alternate}

\usepackage[utf8]{inputenc}
\usepackage[T1]{fontenc}
\usepackage{textcomp}
\usepackage[french, english]{babel}
\usepackage{graphicx}
\usepackage{url}
\usepackage{backnaur}
\usepackage{tikz}
\usepackage{xcolor}
\usepackage{color}
\usepackage{listings}
\usepackage{inconsolata}

\usepackage[hyperref=true,%
            url=false,%
            isbn=false,%
            style=numeric,%
            maxcitenames=3,%
            maxbibnames=100,%
            backend=bibtex,%
            block=none]{biblatex}
\bibliography{references.bib}

\definecolor{red}{rgb}{1,0,0.29}
\definecolor{green}{rgb}{0,0.91,0.75}
\definecolor{lightgray}{rgb}{.9,.9,.9}
\definecolor{gray}{rgb}{.6,.6,.6}
\definecolor{darkgray}{rgb}{.4,.4,.4}

\newcommand{\testbf}{bx}

\newcommand{\CodeSymbol}[1]{
    \bfseries\textcolor{red}{#1}
}

\newcommand\opstyle{\color{red}}

\lstdefinelanguage{js}{
  keywords={break, case, catch, continue, debugger, default, delete, do, else, false, finally, for, function, if, in, instanceof, new, null, return, switch, this, throw, true, try, typeof, var, void, while, with},
  morecomment=[l]{//},
  morecomment=[s]{/*}{*/},
  morestring=[b]',
  morestring=[b]",
  ndkeywords={class, export, boolean, throw, implements, import, this},
  keywordstyle=\color{red}\bfseries,
  ndkeywordstyle=\color{red}\bfseries,
  identifierstyle=\color{black},
  commentstyle=\color{lightgray}\ttfamily,
  stringstyle=\color{gray}\bfseries,
  basicstyle=\color{gray}\ttfamily\scriptsize,
  sensitive=true,
  escapeinside={\/\/\@}{\@}
}

\lstdefinelanguage{flx}{
  keywords={flx, fluxion},
  morecomment=[l]{//},
  morecomment=[s]{/*}{*/},
  morestring=[b]',
  morestring=[b]",
  literate= {+>}{{\ttfamily\scriptsize>>}}2
            {->}{{\CodeSymbol{->}}}2
            {>>}{{\CodeSymbol{>>}}}2, % This is a hack to get -> and >> be highlighted like keywords.
  ndkeywords={main, express\_dispatcher, app\_js, get, handler, reply, anonymous\_1000, null},
  keywordstyle=\color{red}\bfseries,
  ndkeywordstyle=\color{black}\bfseries,
  identifierstyle=\color{black},
  commentstyle=\color{lightgray}\ttfamily,
  stringstyle=\color{gray}\bfseries,
  basicstyle=\color{gray}\ttfamily\scriptsize,
  sensitive=true,
  escapeinside={\/\/\@}{\@}
}

\newcommand{\userlstset}[1]{
  \lstset{ %
    numberstyle=\tiny,               % the size of the fonts that are used for the line-numbers
    numbers=left,                    % where to put the line-numbers
    stepnumber=1,                    % the step between two line-numbers. If it is 1 each line will be numbered
    numbersep=5pt,                   % how far the line-numbers are from the code
    showspaces=false,                % show spaces adding particular underscores
    showstringspaces=false,          % underline spaces within strings
    showtabs=false,                  % show tabs within strings adding particular underscores
    tabsize=2,                       % sets default tabsize to 2 spaces
    captionpos=b,                    % sets the caption-position to bottom
    breaklines=true,                 % sets automatic line breaking
    breakautoindent = true,          %
    breakatwhitespace=false,         % sets if automatic breaks should only happen at whitespace
    escapeinside={\@}{\@},           % if you want to add a comment within your code
    language=#1,                     % choose the language of the code
  } %
}

\newcommand{\ic}[1]{\lstinline|#1|}

\lstnewenvironment{code}[1][js]{%
  \userlstset{#1}%
}{%
}

\newcommand{\includecode}[2]{%
  \userlstset{#1}%
  \lstinputlisting{#2}%
}

\newcommand{\ftnt}[1]{%
\footnote{\small{\url{#1}}}%
}

\newcommand*\circled[1]{%
  \tikz[baseline=(char.base)]{%
    \node[shape=circle,draw,inner sep=0.8pt] (char) {#1};%
  }%
}

\begin{document}

\title{
  Transforming Javascript Event-Loop Into a Pipeline
}

\numberofauthors{2}
\author{
\alignauthor
Etienne Brodu, Stéphane Frénot\\
  \email{\textsf{\normalsize{\{etienne.brodu, stephane.frenot\}@insa-lyon.fr}}}\\
  \affaddr{\textsf{\small{Univ Lyon, INSA Lyon, Inria, CITI, F-69621 Villeurbanne, France}}}
\and
\alignauthor
Frédéric Oblé\\
  \email{\textsf{\normalsize{frederic.oble@worldline.com}}}\\
  \affaddr{\textsf{\small{Worldline, Bât. Le Mirage, 53 avenue Paul Krüger}}}\\
  \affaddr{\textsf{\small{CS 60195, 69624 Villeurbanne Cedex}}}\\
}

\begin{CCSXML}
<ccs2012>
<concept>
<concept_id>10011007.10011006.10011041.10011047</concept_id>
<concept_desc>Software and its engineering~Source code generation</concept_desc>
<concept_significance>500</concept_significance>
</concept>
<concept>
<concept_id>10011007.10011006.10011041.10011048</concept_id>
<concept_desc>Software and its engineering~Runtime environments</concept_desc>
<concept_significance>500</concept_significance>
</concept>
</ccs2012>
\end{CCSXML}

\ccsdesc[500]{Software and its engineering~Source code generation}
\ccsdesc[500]{Software and its engineering~Runtime environments}

% \CopyrightYear{2016} %
% \setcopyright{licensedothergov} %
% \conferenceinfo{SAC 2016,}{April 04 - 08, 2016, Pisa, Italy \vspace{3pt}\\ %
% {\scriptsize Copyright is held by the owner/author(s). Publication rights licensed to ACM.}} %
% \isbn{978-1-4503-3739-7/16/04}\acmPrice{\$15.00} %
% \doi{http://dx.doi.org/10.1145/2851613.2851745} %

\maketitle

\begin{abstract}

The development of a real-time web application often starts with a feature-oriented approach allowing to quickly react to users feedbacks.
However, this approach poorly scales in performance.
Yet, the audience can increase by an order of magnitude in a matter of hours. This first approach is unable to deal with the higher connections spikes.
It leads the development team to shift to a scalable approach often linked to new development paradigm such as dataflow programming.
This shift of technology is disruptive and continuity-threatening.
To avoid this shift, we propose to abstract the feature-oriented development into a more scalable high-level language.
Indeed, reasoning on this high-level language allows to dynamically cope with audience growth and decrease.

We propose a compilation approach that transforms a Javascript, single-threaded real-time web application into a network of small independent parts communicating by message streams.
We named these parts \textit{fluxions}, by contraction between a flow (flux in french) and a function.
The independence of these parts allows their execution to be parallel, and to organize an application on several processors to cope with its load, in a similar way network routers do with IP traffic.
% The dynamic reorganization of these parts in a cluster of machine can help an application to deal with its load in a similar way network routers do with IP traffic.
% \comment{it might be true, but it seems completely unrelated to me. Maybe because I don't understand IP enough}
We test this approach by applying the compiler to a real web application.
We transform this application to parallelize the execution of an independent part and present the result.

\end{abstract}

\endinput

% The very short abstract 50 words

Web applications starts with a feature-oriented approach to quickly react to users feedbacks, but eventually shift to a more scalable approach.
To avoid this disruptive shift, we propose an equivalence to compile the feature-oriented approach to a scalable high-level language.
We test this approach by applying the compiler to a real web application.

\printccsdesc
\keywords{Flow programming; Web; Javascript}

\end{document}